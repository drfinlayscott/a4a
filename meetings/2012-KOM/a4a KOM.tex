%% LyX 2.0.2 created this file.  For more info, see http://www.lyx.org/.
%% Do not edit unless you really know what you are doing.
\documentclass[english]{article}
\usepackage[T1]{fontenc}
\usepackage[utf8]{inputenc}
\usepackage[a4paper]{geometry}
\geometry{verbose,tmargin=3cm,bmargin=2cm,lmargin=2cm,rmargin=2cm}
\setlength{\parskip}{\medskipamount}
\setlength{\parindent}{0pt}
\usepackage{color}
\usepackage{amsthm}
\usepackage{amsmath}
\usepackage{amssymb}
\usepackage{setspace}
\onehalfspacing

\makeatletter

%%%%%%%%%%%%%%%%%%%%%%%%%%%%%% LyX specific LaTeX commands.
%% Because html converters don't know tabularnewline
\providecommand{\tabularnewline}{\\}

%%%%%%%%%%%%%%%%%%%%%%%%%%%%%% Textclass specific LaTeX commands.
\numberwithin{equation}{section}
\numberwithin{figure}{section}

\@ifundefined{date}{}{\date{}}
%%%%%%%%%%%%%%%%%%%%%%%%%%%%%% User specified LaTeX commands.
\usepackage[english]{babel}
\usepackage{amsfonts}
\usepackage{array}
\usepackage{hhline}

\makeatother

\usepackage{babel}
\begin{document}

\title{a4a kick-off meeting report}

\maketitle

\section{Introduction}

The implementation of the 2009 revision of the DCF generated the obligation
to collect a large amount of information for all stocks being subject
to fisheries exploitation. Most of these stocks will have in the future,
$\sim$2020, time series of exploitation data more then 10 years long,
although the information about the biology will most likely be limited
due to the high requirements of man power and logistics to process
all the samples collected. 

These stocks (will) have a moderate amount of information and don't
fit into the \textquotedbl{}data poor\textquotedbl{} stock definition,
neither will allow runing sophisticated modelling methods. Having
stock assessment and adisory methods to apply to a large number of
moderate data stocks, raise interesting challenges and creates opportunities
worth exploring. For example, approaching stock assessment as a data
generating engine, having a common stock assessment methodology or
analysing massive stock assessment results, open the possibility of
issuing advise for more species in a multifleet, multispecies framework
and promotes comparative advise analysis. 

As scientists it is important to think ahead and start developing
such methodologies. JRC following it's mission of antecipating policy
implementation issues decided to move forward with the ``Assessment
for All'' (a4a) initiative, aiming to: 
\begin{enumerate}
\item develop an assessment method targeting stocks that have a reduced
knowledge base on biology and moderate time series on exploitation
and abundance; 
\item trigger the discussion about the problem of massive stock assessment;
\item build capacity on stock assessment and fisheries management advise.
\end{enumerate}
The initiative has a web repository where all the presentations, reports,
code and data can be found%
\footnote{https://github.com/ejardim/a4a%
}.


\subsection{Objectives and organization}

The kick-off meeting of the initiative took place between 29/February
and 02/March of 2012 at the JRC head quarters in Varese, Italy, chaired
by Ernesto Jardim (JRC/EC), with the following terms of reference:
\begin{enumerate}
\item world wide needs of fisheries advice; 
\item available stock assessment methods; 
\item ways to make stock assessment in the context of management advise
more robust; 
\item identify modules to build a MSE; 
\item discuss progress of the initiative. 
\end{enumerate}
and the following agenda:
\begin{itemize}
\item Wednesday

\begin{itemize}
\item Introduction (E.Jardim)
\item Identify and describe the problem
\item Summary of WKLIFE (M.Azevedo)
\item Summary of south hemisphere initiative report (I.Mosqueira)
\item Summary of GFCM WK on Elasmobranchs (G.Chato)
\item Management of a selection of stocks from N.America (A.Cooper)
\item Discussion
\item Compile a set of possible solutions to the problem
\end{itemize}
\item Thursday

\begin{itemize}
\item Seminar to JRC on Fisheries Modelling (09:30 – 11:00)

\begin{itemize}
\item Welcome message by Alessandra Zampieri (JRC HoU)
\item MSE by Jose de Oliveira (CEFAS)
\item Catch Dynamic Model by Ruben Roa (AZTI)
\item Genetics and quantitative fisheries by Gary Carvalho (B.U.) \& Heiner
Nielsen (DTU-AQUA)
\end{itemize}
\item Elaborate on advantages and disadvantages of each solution
\item Revisit the solutions and decide which are the most promising
\item Agree on a framework for testing: MSE, statistical analysis, simulated
data, etc.
\end{itemize}
\item Friday

\begin{itemize}
\item Discuss implementation and testing of the best solutions
\item Elaborate on the expected outcome
\item Challenges and opportunities
\item Workplan
\end{itemize}
\end{itemize}

\subsection{Participation}

\begin{tabular}{rrl}
\hline 
Name &  & Affiliation\tabularnewline
\hline 
Andrew Cooper &  & Simon Fraser University (CA)\tabularnewline
Chato Osio &  & Joint Research Center (EC)\tabularnewline
Einar Nielsen &  & Danish Technical University (DK)\tabularnewline
Ernesto Jardim (chair) &  & Joint Research Center (EC)\tabularnewline
Finlay Scott &  & Centre for Environment, Fisheries \& Aquaculture Science (UK)\tabularnewline
Gary Carvalho &  & Bangor University (UK)\tabularnewline
Iago Mosqueira &  & Joint Research Center (EC)\tabularnewline
Jann Marthinson &  & Joint Research Center (EC)\tabularnewline
Jose de Oliveira &  & Centre for Environment, Fisheries \& Aquaculture Science (UK)\tabularnewline
Leire Ibarriaga &  & AZTI Technalia (SP)\tabularnewline
Manuela Azevedo &  & Portuguese Institute for Sea and Atmosphere (PT)\tabularnewline
Ruben Roa &  & AZTI Technalia (SP)\tabularnewline
\hline 
\end{tabular}


\section{Outcomes}

The ToR and agenda were followed loosely to allow for discussions
and brain storming. 

The overall objective was to consolidate ideas regarding the initiative's
aims, expectations and opeartionalization. It was critical to better
define the problem the initiative is attempting to contribute to,
discuss the range of solutions available and define which ones the
initiative should pursue. In that sense the meeting was succefull
and the objectives were achieved. 

The discussions taken place gave raise to the definition of ``moderate
data stock'', a major step to understand where the initiative pretends
to contribute. The issue of introducing genetics in fisheries management
was largely discussed and several clarifications were possible, as
well as the identification of promising areas of convergence between
fisheries modelling and genetics. Another issue to be grounded was
the assessment model(s) to be explored, for which a set of characteristics
were defined but leaving open the exact framework to be used. A step
further was taken by designing the type of Management Strategies Evaluation
the initiative will promote/develop as a standard advise methodology
for these kind of stocks. In a more pragmatic setting the group loosely
defined a simulation experiment to be carried out and a set of subjects
to be tested. Additionaly the group discussed the a4a operationalization.


\subsection{Moderate data stock}

The group discussed the characteristics of a ``moderate data stock''
in terms of data availability and concluded on the following definition.

A moderate data stock has at least data on:
\begin{itemize}
\item nominal effort, 
\item volume of catches in weight (which should include landings and discards), 
\item length structure of the catches (which should include landings and
discards),
\item information based maturity ogive (parameters are not educated guesses
but based on information),
\item information based growth model (parameters are not educated guesses
but based on information),
\item length-weight relationship,
\item index of abundance (the type of index is left open on purpose, it
could be a survey or CPUE),
\item length information for the index of abundance (based on selectivity
studies or direct observations)
\end{itemize}
The length of the time series is not specified on pourpose, it will
depend on the species' longevity, however it was generally accepted
as a rule of thumb that it should cover at least one cohort {[}\textcolor{red}{!!==>\textcompwordmark{}>
CHECK THIS, IT WAS NOT CLEARLY DISCUSSED DURING THE MEETING <\textcompwordmark{}<==!!}{]}. 


\section{Genetics}
\begin{itemize}
\item population size

\begin{itemize}
\item effective population size

\begin{itemize}
\item Doesn't work well at yearly time scales 
\end{itemize}
\end{itemize}
\item frequency of rare alleles 
\item May work for endareged species 
\item genetic diversity

\begin{itemize}
\item monitor specifi genes that are linked to specific biological functions 
\end{itemize}
\item indicators are still difficult to id and measure 
\item May work for endareged species 
\item reproductive success 
\item IUU

\begin{itemize}
\item 30-50\% scale 
\end{itemize}
\item markers easy to compare between labs and reproducible 
\item Monitor over time 
\item Stock ID

\begin{itemize}
\item inout rates

\begin{itemize}
\item requires baselines 
\end{itemize}
\item \% of individuals 
\end{itemize}
\item Id biological representative characteristics 
\item risk indicator

\begin{itemize}
\item time evolution of distribution may give indications of something \textquotedbl{}wrong\textquotedbl{}
going on 
\end{itemize}
\item monitor specifi genes that are linked to specific biological functions 
\item mixt stock analysis

\begin{itemize}
\item helps borrowing information between stocks 
\end{itemize}
\item rate of change of \textquotedbl{}fast\textquotedbl{} allele frequency 
\item MSE

\begin{itemize}
\item What if we're exploring two pops with distinct biological characteristics

\begin{itemize}
\item that are id by genetics 
\end{itemize}
\item \%pop = HCR 
\end{itemize}
\item how robust if my HCR to breaking stock assessment assunmptions of
closed population

\begin{itemize}
\item migration into fishery 
\item migration into the SSB 
\end{itemize}
\end{itemize}

\section{Biomass model/stock assessment model}
\begin{itemize}
\item Unscaled 
\item Scaled

\begin{itemize}
\item Model to be applied rapidly to a wide range of situations 
\item Results must be used for advise on a quantitative basis 
\item surplus to get biomass and feed to length structure 
\end{itemize}
\item Referenced 
\end{itemize}

\section{Assessment/advise methodology}
\begin{itemize}
\item data moderate stock 
\item Standard procedures to test assumptions and conditions and give advise 
\item OM

\begin{itemize}
\item Biomass model/stock assessment model

\begin{itemize}
\item Unscaled 
\item Scaled

\begin{itemize}
\item Model to be applied rapidly to a wide range of situations 
\item Results must be used for advise on a quantitative basis 
\item surplus to get biomass and feed to length structure 
\end{itemize}
\item Referenced 
\end{itemize}
\item A set of LH \& other bio/gen relevant factors

\begin{itemize}
\item Conditioning using LH to fill gaps and introduce variability on the
OM 
\item Set a protocol to define the range of OM based on LH (e.g. LH+ 1 sd
etc) 
\end{itemize}
\end{itemize}
\item MP

\begin{itemize}
\item A set of metric generators to HCR 
\item A set of HCR

\begin{itemize}
\item Generic, multi-element, switch on/off 
\end{itemize}
\end{itemize}
\item Which elements shoud constitute the standard 
\item Presenting results 
\item Statistic to include the uncertainty of the system

\begin{itemize}
\item B\_\{obs\}-B\_\{true\} \\
accounts for observation and estimation error 
\item (C\_\{after implementation\}/C\_\{from hcr\}) / (B\_\{true\}/B\_\{obs\})
\\
it could be interpreted as a measure of the trade-off between implementation
error and observation-estimation error. Being close to 1 could be
interpreted as errors being compensated and being able to manage “well”
the system. The clear disadvantage is that it is not symmetric: goes
in (0,1) in one side and (1,infinity) on the other.
\end{itemize}
\end{itemize}

\section{Simulation experiment}
\begin{itemize}
\item Simulate data that fits the definition of moderate data

\begin{itemize}
\item Ask Daniel Howell for spatial data 
\item gislasim 
\item NOAA toolbox 
\item Link with WGMG 
\end{itemize}
\item Use real data that fits the definition of moderate data

\begin{itemize}
\item conditional simulation of pop and fisheries and test assessment methods 
\end{itemize}
\item Id relevant elements that are subject for testing 
\item Write MSE using the ICES HCR 
\item Expand to more HCR

\begin{itemize}
\item HCR with parameter depending on objective 
\end{itemize}
\item Tests

\begin{itemize}
\item How the length of the time series impacts the results 
\item How violation of assumptions or fixed parameters impact the results 
\item What if we're exploring two pops with distinct biological characteristics
that are id by genetics 
\item \%pop = HCR 
\item how robust if my HCR to breaking stock assessment assunmptions of
closed population
\item Test model complexity, does a BioDyn model works better than a structured
model

\begin{itemize}
\item Will depend on the reliability of the abundance index 
\end{itemize}
\end{itemize}
\end{itemize}

\section{Operationalise}
\begin{itemize}
\item FLR 
\item short time visits 
\item scientific meetings 
\item experts have some time for dev. and review during next year \end{itemize}

\end{document}
